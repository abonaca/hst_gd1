%%%%%%%%%%%%%%%%%%%%%%%%%%%%%%%%%%%%%%%%%%%%%%%%%%%%%%%%%%%%%%%%%%%%%%%%%%
%
%    phase1-GO.tex  (use only for General Observer and Snapshot proposals; use phase1-AR.tex for Archival Research and
%                      Theory proposals use phase1-DD.tex for GO/DD proposals and use phase1-MC.tex for GO/MC rapid response
%                       proposals).
%
%    HUBBLE SPACE TELESCOPE
%    PHASE I OBSERVING PROPOSAL TEMPLATE 
%     FOR CYCLE 25 and beyond
%
%    Version 2.0 April 2018 
%
%    Guidelines and assistance
%    =========================
%     Cycle 26 Announcement Web Page:
%
%         https://hst-docs.stsci.edu/ 
%
%    Please contact the STScI Help Desk if you need assistance with any
%    aspect of proposing for and using HST. Either send e-mail to
%    help@stsci.edu, or call 1-800-544-8125; from outside the United
%    States, call [1] 410-338-1082.
%
%
%%%%%%%%%%%%%%%%%%%%%%%%%%%%%%%%%%%%%%%%%%%%%%%%%%%%%%%%%%%%%%%%%%%%%%%%%%%

% The template begins here. Please do not modify the font size from 12 point.

\documentclass[12pt]{article}
\usepackage{phase1}
\renewcommand{\floatpagefraction}{.85}

\usepackage{graphicx}
\usepackage{parskip} \setlength{\parindent}{0cm}

\usepackage{titlesec}
\titlespacing*{\section}{0pt}{0pt}{-3pt}
\titlespacing*{\subsection}{0pt}{0pt}{-3pt}

\usepackage[backend=biber, style=authoryear-comp, doi=false, isbn=false, url=false, giveninits=false, minbibnames=1, maxbibnames=2, maxcitenames=2, uniquelist=false]{biblatex}
\usepackage{bib4apps}
\addbibresource[]{gd1.bib}

\newcommand{\hst}{\textsl{HST}}
\newcommand{\gaia}{\textsl{Gaia}}

\begin{document}

%   1. SCIENTIFIC JUSTIFICATION
%       (See https://hst-docs.stsci.edu/display/HSP/HST+Cycle+26+Preparation+of+the+PDF+Attachment)
%
%
\justification          % Do not delete this command.
\vspace{-0.5cm}

Clumpiness of matter on small scales remains one of the most pressing questions in cosmology and galaxy formation \parencite{bullock2017}.
In the $\Lambda$CDM model, clumps of dark matter with no baryons are expected to exist with masses as low as $10^{6}\rm\,M_\odot$ \parencite{springel2008}, while alternative models have a higher cut-off in the matter power spectrum \parencite[e.g.,][]{bode2001,hu2000}.
Cold stellar streams --- remnants of disrupted globular clusters discovered in the Galactic halo \parencite{gc2016} --- are unique and powerful tools to study the properties of mass and dark matter within the Galaxy \parencite[e.g.,][]{bonaca2018}.
In particular, an encounter between a stellar stream and a dark matter subhalo perturbs the orderly structure of the stellar stream, produces density variations along the stream \parencite[e.g.,][]{carlberg2012}, and, depending on geometry, possibly also folds a part of the stream \parencite[e.g.,][]{yoon2011}.
{\bf We propose to use \hst\ to measure high-precision proper motions of the GD-1 stellar stream, a nearby, dense stellar stream that shows the signature of a recent encounter with a dark matter subhalo.}
Definitive signatures of such an interaction would provide critical tests of dark matter theories.

\section*{Background: Testing the $\Lambda$CDM cosmology on small scales}
Upon inception, theoretical predictions of the $\Lambda$CDM universe met with favorable comparison to observations: galaxies grow hierarchically and they are clustered on large scales.
With the increase in resolution came discrepancies, chief of which is that simulated $\Lambda$CDM galaxies have too many satellites.
Within the dark matter cosmological framework, there are two classes of solutions within that aim to reconcile these observations: either (1) small halos of dark matter do not exist (implying an ultra-light or a moderately massive dark matter particle), or (2) they exist, but have escaped our detection because they host no stars (and dark matter particle is very massive).
If these small halos exist, they still exert gravitational influence, so a number of searches for dark matter structure on small scales have been launched to distinguish between these scenarios.

\subsection*{Extragalactic searches for small-scale structure}
Gravitational lensing is a promising way to search for dark matter subhalos in a cosmological setting.
- lensing anomalies mapped to subhalos
As an ultimate test of the CDM paradigm, lensing can be applied to a number of systems, but only massive objects have been detected so far.
Objects of this mass in the local universe host stars, so increases in sensitivity are sought after.

\subsection*{Subhalo searches in the Milky Way using tidal streams}
In the local universe, counts of density gaps in thin stellar streams have been mapped to the abundance of dark matter subhalos down to $\sim10^6\,\rm M_\odot$ \parencite[e.g.,][]{carlberg2012,carlberg2013b}.
In addition to dark matter subhalos, a number of physical processes can perturb streams at this level \parencite[e.g.,][]{kupper2008, amorisco2016}.
As a result, no stream has definitively established the presence of dark matter substructure.
However, thanks to a recent \gaia\ discovery of a perturbation site in the GD-1 stream, we now have a prime target for follow-up.

\begin{figure}
\begin{center}
\includegraphics[width=\textwidth]{fields.pdf}
\end{center}
\caption{
Likely GD-1 stream members in the stream reference frame, identified based on \gaia\ proper motions and position in the CMD \parencite{pwb2018}.
Density variations are immediately apparent in this cleanest view of the stream to date.
{\bf We propose to determine the origin of the stream gap} at $\phi_1\approx-40^\circ$ by using \hst\ proper motions measured in fields on both sides of the gap and in the spur extending from the gap (orange circles).
}
\label{fig:fields}
\end{figure}

\section*{This proposal: Clumpiness of the Milky Way halo with the GD-1 stream}
% - both Galactic and extragalactic results exciting, but lacking specifics
% - here we have an opportunity to get extremely specific

\subsection*{GD-1 -- a halo stream with signatures of perturbation}
The GD-1 stellar stream was discovered using photometry from the Sloan Digital Sky Survey \parencite{grillmair2006},
and, in combination with a handful of radial velocities for red giant star members, has been used to measure the mass and shape of the large-scale mass distribution within the Galaxy \parencite{koposov2010, bovy2016}.
Using deeper photometry, \textcite{deboer2018} noted density variations along the stream and possible deviations of the main stream track that are not expected from simple models of the stream formation.
However, photometric studies alone suffer from significant contamination from background and foreground stars.
As an alternative approach to selecting likely GD-1 members, \textcite{pwb2018} mapped the stream additionally selecting members using the exquisite proper motions from the \gaia\ mission (Figure~\ref{fig:fields}).
This relatively contamination-free view of the stream reveals two significant gaps along the stream.
The stellar density in one of the gaps ($\phi_1\sim-20^\circ$) is at the level of remaining contamination, and possibly the location where the GD-1 progenitor completely disrupted (see Figure~2 in \cite{pwb2018}).
The other gap ($\phi_1\sim-40^\circ$) is associated with a large stream overdensity ($\phi_1\lesssim-40^\circ$) and a spur of member stars ($\phi_1\gtrsim-40^\circ$) parallel to the main stream, but offset from it by $\sim1^\circ$.
Such morphology can result from a stream interaction with a massive perturber (Figure~\ref{fig:fiducial}).

\begin{figure}
\begin{center}
\includegraphics[width=0.9\textwidth]{pm_forecasts.pdf}
\end{center}
\caption{
(\emph{a}) A simulated GD-1-like stream interacts with a $3\times10^7\,\rm M_\odot$ perturber and after 0.5\,Gyr develops qualitative features observed in the GD-1 stream: a gap in the stream at the location of the closest approach, a loop of stars extending on the one side of the gap as a spur off from the main stream, and producing an overdensity on the other side of the gap.
% (\emph{Top}) An interaction of a GD-1-like stream with a $3\times10^{7\,\rm M_\odot}$ perturber after 0.5\,Gyr develops qualitative features observed in the GD-1 stream: a gap in the stream at the location of closest approach, a loop of stars extending on the one side of the gap as a spur off from the main stream, and producing an overdensity on the other side of the gap.
(\emph{b}) Stars in the loop are kinematically offset from the main stream.
The sign of the offset is opposite on either side of the gap in at least one component of the tangential velocity, and a unique signature of interaction.
(\emph{c}) Proper motions of the simulated GD-1 convolved with the expected end-of-mission \gaia\ uncertainties.
(\emph{d}) Proper motions of the simulated GD-1 convolved with the expected \hst\ uncertainties.
{\bf The interaction signature is evident in proper motions at \hst\ precision}, but unresolved even with the final \gaia\ data.
}
\label{fig:fiducial}
\end{figure}

\subsection*{Proper motion offsets as a unique feature of a perturbed stream}
A massive perturber passing quickly by a thin stellar stream imparts a velocity kick to stars closest to the encounter \parencite[e.g.,][]{erkal2015}.
As a result, a loop of stars is pulled off from their original orbits along the stream, thus opening a gap in the stream.
Typically, the loop is aligned with the stream on both sides of the gap, producing a characteristic density profile \parencite[e.g.,][]{carlberg2013}.
At certain impact geometries and viewing angles, however, one side of the loop can appear to extend from the gap, while being aligned with the stream on the opposite side (Figure~\ref{fig:fiducial}\,a).
This conceptual model qualitatively matches features in GD-1, and has a distinct kinematic signature.
As the loop continues to expand, it has a relative velocity with respect to the adjacent stream stars (Figure~\ref{fig:fiducial}\,b).
Uniquely, the velocity offset between stars in the loop and the main stream has the opposite sign on either side of the gap in at least one component of the tangential velocity.

\begin{figure}
\begin{center}
\includegraphics[width=0.8\textwidth]{lethe.jpg}
\end{center}
\caption{
A proper motion analysis of a field in the Lethe stream from \textcite{sohn2016}.
Much like GD-1, Lethe is a thin, nearby stream and its members (circled data points) have been identified as following an old and metal-poor isochrone placed at 12\,kpc in the color-magnitude diagram (\emph{left}) that are kinematically cold in proper motions (\emph{right}).
Lethe is less prominent than GD-1, and thus sets a lower limit on the signal expected from GD-1 observed using a similar observational setup.
}
\label{fig:lethe}
\end{figure}

{\bf The key goal of this proposal is to measure tangential velocities in the perturbed region of GD-1 and establish the magnitude of velocity offsets between different locations.}
Scenarios that produce GD-1-like morphology have the spur moving with a relative velocity of $\sim10\,\rm km\,s^{-1}$ with respect to the stream.
In order to differentiate these features kinematically at the distance of GD-1, a proper motion precision of $\lesssim0.1\,\rm mas\,yr^{-1}$ is required (Figure~\ref{fig:fiducial}\,d).
\hst\ astrometry at such precision has already been demonstrated for a thin stellar stream, Lethe \parencite[Figure~\ref{fig:lethe}, $\sigma_\mu\sim0.05\,\rm mas\,yr^{-1}$;][]{sohn2016}.
Taking into account different time baselines (10\,yrs for Lethe and 3\,yrs for GD-1) and stream densities \parencite[GD-1 is $\sim$ twice as dense;][]{grillmair2006, grillmair2009}, we estimate that the required precision can be reached by averaging over GD-1 stars in two \hst\ fields.

% Although most of the dark matter subhalos in the inner Galaxy are disrupted by the disk \parencite{gk2017}, 
Accounting for the disruption of subhalos by the Galactic disk \parencite{gk2017}, GD-1 is expected to have encountered $1-2$ massive subhalos in the past $\sim7$\,Gyr \parencite{erkal2016}.
If the measured velocity offsets confirm that the gap at $\phi\sim-40^\circ$ was caused by a subhalo impact, their magnitude and direction will further constrain the geometry of the encounter.
% Combined with radial velocity data, from our ongoing follow-up, as well as public surveys \parencite[e.g.,][]{}, .
With the details of a known encounter in hand, the current location of the perturber will be uncertain only due to the mass-velocity degeneracy.
Such a prediction might provide an exciting search volume for direct observations of subhalos via weak lensing \parencite{vtilburg2018}.
Alternatively, establishing that the prime GD-1 gap is not due to an encounter will put a stringent upper limit on the abundance of massive subhalos in the inner Galaxy.

\subsection*{Ancillary tests of dark matter}
While the main goal of this proposal is to establish whether GD-1 was perturbed by a dark matter subhalo, the resulting data can be used to further test the cosmological paradigm.
Here are some of the most exciting directions we will pursue.

\emph{Absolute GD-1 motion:}
GD-1 is the longest thin stream in the Milky Way, and as such it is also extremely informative about the distribution of matter on the galactic scale \parencite{bonaca2018}.
Modeling of GD-1, and all other thin streams, has been limited to simplistic models of the Galaxy \parencite[e.g.,][]{koposov2010}, although studies of the more massive Sagittarius stream have shown the need for more flexible models \parencite[e.g.,][]{ibata2013,fardal2018}.
Following this work, the absolute proper motions of GD-1 will be known to an unprecedented precision.
These high-quality data will permit the study of a thin stream in realistic models of the Galaxy for the first time.

\emph{Velocity dispersion in GD-1:}
- limits for total heating, either from subhalos, or fuzzy dark matter
- even with a few stars, after deconvolving observational uncertainties, can put limits on the intrinsic velocity dispersion, as in \textcite{sohn2016}

\emph{Kinematic structure of the stellar halo:}
- follow up on \parencite{deason2013}

\section*{Why \hst\ is needed}
This proposal requires proper motions of faint stars precise to $\lesssim0.1\,\rm mas\, yr^{-1}$.
Even though \gaia\ astrometry continues to revolutionize the field of Galactic dynamics, its expected end-of-mission performance is $\sim0.25\,\rm mas\,yr^{-1}$ on the faint end (Figure~\ref{fig:fiducial}\,d).
Proper motions from the ground are becoming more precise with the advent of adaptive optics, but are still not competitive to \hst-only measurements, either in terms of precision or field-of-view \parencite{fritz2017}.
At the time being, {\bf \hst\ remains the only facility capable of delivering high-precision proper motions for faint, distant stars}.

%%%%%%%%%%%%%%%%%%%%%%%%%%%%%%%%%%%%%%%%%%%%%%%%%%%%%%%%%%%%%%%%%%%%%%%%%%%

%   2. DESCRIPTION OF THE OBSERVATIONS
%       (See https://hst-docs.stsci.edu/display/HSP/HST+Cycle+26+Preparation+of+the+PDF+Attachment)
%
%
\describeobservations   % Do not delete this command.
\vspace{-0.5cm}
\section*{Astrometry with \hst}
Even in the era of \gaia, \hst\ has unparalleled astrometric capabilities for faint and distant targets.
Source displacements can be measured over time to produce extremely accurate proper motions thanks to the well established geometric distortion solutions, PSF calibrations, and measurement techniques \parencite[e.g.,][]{anderson2003}.
Through the past decade, \hst\ has been used to study internal dynamics in the Milky Way satellites by measuring relative motions \parencite[e.g., ][]{anderson2010} and also to measure absolute motions of these objects as they orbit the Milky Way.
Quasars have conventionally been used as stationary reference sources in HST PM studies of satellite galaxies \parencite[e.g.,][]{kallivayalil2006, kallivayalil2013, piatek2008}, and this has revolutionized our understanding of the Magellanic Clouds \parencite{besla2007}.
However, quasars are sparsely distributed on the sky, and there is generally at most one quasar in an \hst\ field.
Often, a field of interest contains no known quasars.

Recently, a new technique of absolute astrometry with \hst\ has been developed using distant background galaxies as reference sources \parencite[e.g.,][]{mahmud2008, sohn2012,sohn2013}.
There are many (typically $N\sim100$) astrometrically useful background galaxies in any deep \hst\ image.
This provides two advantages over quasars.
First, by averaging one obtains a $\sqrt{N}$ improvement in astrometric accuracy, even if any individual background galaxy can not be centroided as accurately as a point source (quasar).
Background galaxies also sample a large part of the detector, which reduces systematic errors.
Second, by using background galaxies it is possible to perform absolute astrometry on any field with deep multi-epoch \hst\ imaging.
It is the combination of these two advantages that makes the project we propose here possible.
This technique has already been demonstrated to yield proper motions of galaxies accurate to $\sim0.01-0.03\,\rm mas\,yr^{-1}$ (M31, \cite{sohn2012,vdmarel2012a,vdmarel2012b} and Leo~I, \cite{sohn2013}), the motions of a small sample of distant stars in the Milky Way halo \parencite{deason2013}, the motions of stars in the Sagittarius stream \parencite{sohn2015}, and, particularly pertaining to this project, the motions of stars in cold streams such as Lethe \parencite[Figure~\ref{fig:lethe},][]{sohn2016}.


\section*{Target fields and stellar density}
In hopes of leveraging existing data, yielding large time baselines, and minimizing the need for new orbits, we searched the \hst\ archive for existing deep imaging serendipitously located on the GD-1 stream features reported in \textcite{pwb2018}.
Unfortunately, no suitable fields were identified for use as first epoch for our astrometric study, so we selected 3 new fields to be observed in two epochs: two on either side of the GD-1 gap located at $\phi_1\approx-40^\circ$, and one on the associated spur.
The fields are shown in Figure~\ref{fig:fields}.
% and listed in Table~\ref{t:fields}.

\begin{figure}
\begin{center}
\includegraphics[width=\textwidth]{target_density.pdf}
\end{center}
\caption{
(\emph{Left}) Color-magnitude diagram of the GD-1 region in PanSTARRS bands (small gray points) and likely GD-1 members (large blue points), selected on Gaia proper motions and PS1 photometry as in Price-Whelan \& Bonaca (2018).
An old, metal-poor isochrone fits GD-1 at a distance of 8\,kpc (light blue line).
(\emph{Right}) Predicted number of stars in combined ACS/WFC and WFC3/UVIS fields-of-view as a function of magnitude (light blue line), scaled from the average density of GD-1 members on the bright end (dark blue line).
There are $\sim15$ member stars expected at $g\lesssim26$\,mag, which is observed at $\rm S/N\sim20$ in our setup.
Combining 2--4 fields, depending on the local stream density, \textbf{this project will yield GD-1 proper motions accurate to $\sim3\,\rm km\,s^{-1}$}.
}
\label{fig:density}
\end{figure}

We estimated the expected number of stream stars in each field as follows.
We adopt an isochrone with age, metallicity, and distance from \textcite{koposov2010}, and generate a corresponding luminosity function in the PanSTARRS1 g band \parencite[Figure~\ref{fig:density},][]{dotter2008}.
We then analyze the GD-1 sample from \textcite{pwb2018}, and find the total number of stars selected on \gaia\ proper motions and PanSTARRS photometry along the whole stream.
We choose this large area rather than one that matches the ACS and WFC3 fields to avoid small number statistics in \gaia.
We then use the cumulative distribution of these stars to normalize the luminosity function between $g\approx18$ and $g\approx20$ (where the \gaia\ sample is complete, marked by a solid, dark blue line in Figure~\ref{fig:density}), and then plot the corresponding number of GD-1 stars in a combined ACS/WFC and WFC3/UVIS field of view as a function of magnitude (light blue line).
This tells us the number of GD-1 stream stars we will detect in a pointing down to a given apparent magnitude.
To yield a sufficient number of stars for precise photometry, we will observe 2 pointings on the stream fields.
The stellar density in the spur is a factor of $\approx3$ lower than in the main stream \parencite[compare Figure~5 in][]{pwb2018}, so to compensate, we will observe 4 pointings on the spur field.

\section*{Observing strategy}
We will observe each ACS field in the F606W band for 3 orbits in the first epoch and 3 orbits in the second epoch to do the astrometry.
The number of orbits per epoch is dictated by the need to construct high S/N templates of a sufficient number of background galaxies, and has been empirically established \parencite[e.g.,][]{sohn2012}. 
We will also observe for 1 orbit in F814W, so that we can construct a color-magnitude diagram for each field.
We will select GD-1 stars for the final analysis (and reject foreground stars and other objects) by selecting stars in both CMD and proper motion space (compare Figure~\ref{fig:lethe}).
The colors need not be as accurate as the astrometry, so 1 orbit in F814W suffices.
All individual exposures will be sub-pixel dithered and will last half an orbit or less.
We also request coordinated parallel observations with WFC3, which will also point at the GD-1 stream.

\section*{Expected accuracy}
There are $N=100$ astrometrically useful background galaxies per \hst\ field \parencite[e.g.,][]{sohn2012,sohn2013}, and these can be centroided to an average of $\sim0.04$ pixels per galaxy, per half-orbit exposure.
The position of an individual star can be determined to $\sim2(S/N)^{-1}$ pixels.
With these estimates, it is possible to estimate the predicted proper motion (or velocity) accuracy as a function of stellar magnitude, given the known exposure times and time baselines.
We find that proper motion accuracies remain useful down to apparent g-band magnitude $\sim26$, where the velocity error $\Delta V\approx15\,\rm km\,s^{-1}$ at the distance of GD-1.
Figure~\ref{fig:density} shows that we expect $N\approx15$ stars down to this limit in a combined ACS and WFC3 pointing, or a total of $N\approx30$ stars from 2 pointings in the stream fields, and $N\approx20$ stars from 4 pointings in the more diffuse spur field. 
So the mean transverse velocity can be determined to a \emph{random} error of $\Delta V/\sqrt{N}\lesssim3\,\rm km\,s^{-1}$.
The precision required to differentiate tangential motions in different GD-1 fields is $\sim5\,\rm km\,s^{-1}$, so further degradation in the \hst\ performance can be tolerated.
 
\textcite{sohn2012, sohn2013} have shown that any \emph{systematic} proper motion errors can be controlled to $\lesssim0.04\,\rm mas\,yr^{-1}$ (i.e. $\lesssim2\,\rm km\,s^{-1}$ at the distance of the GD-1 stream).
The ACS/WFC geometric distortion calibration is stable with time to $<0.005$\,pixels \parencite{anderson2007}.
The impact of imperfect CCD CTE on ACS/WFC images will be corrected using a code that applies a time-dependent pixel-by-pixel correction directly to the flat-fielded images \parencite{andersonbedin2010}.
Also, we will fit for PSF variations between epochs directly as part of our astrometric solutions.

\section*{Measurement strategy}
To measure the movement of stars with respect to background galaxies, we will start by creating a co-added frame based on all individual exposures of a field (similar to MultiDrizzle but better optimized for astrometry).
SExtractor will then be run on the co-added frame to detect and classify all sources into point and extended sources.
For the stars in the field, an ePSF \parencite[effective Point Spread Function;][]{anderson2003} will be created.
Similarly, for each individual compact background galaxy a ``GSF'' will be created (a template of the galaxy that takes into account the galaxy morphology, the PSF, and the pixel binning).
For each individual exposure in each epoch, we will fit all stars with the ePSF and all galaxies with their GSF.
The inferred positions will be corrected to a geometrically undistorted frame using the available distortion solutions \parencite{anderson2004} and adopting six-parameter linear transformations to correct for any time-dependent linear skew variations \parencite{anderson2007}.
We will then measure the displacement between the two epochs of the stars with respect to the background galaxies (using only closeby galaxies on the CCD of similar brightness, to minimize systematic errors due to CTE or geometric distortion residuals).
The average of this quantity over all stream stars and all exposures of a given field gives the bulk proper motion of the stream in that particular field.


%%%%%%%%%%%%%%%%%%%%%%%%%%%%%%%%%%%%%%%%%%%%%%%%%%%%%%%%%%%%%%%%%%%%%%%%%%%

%   3. SPECIAL REQUIREMENTS
%       (See https://hst-docs.stsci.edu/display/HSP/HST+Cycle+26+Preparation+of+the+PDF+Attachment)
%
%
\specialreq             % Do not delete this command.
% Justify your special requirements here, if any.
\vspace{-0.5cm}
We are requesting coordinated parallel observations with WFC3.
GD-1 stream is wide enough to allow for simultaneous targeting with ACS and WFC3 ($\rm FWHM\sim0.5^\circ$, Grillmair \& Dionatos 2006), and we will provide orientations to maximally recover the stream.
These parallel observations will increase the number of observed GD-1 members by 60\,\% and thus improve measured velocities by 20\,\%.

For each field we request the same telescope orientation for both epochs.
This will maximize the overlapping area between the two epochs, will minimize differential geometric distortion correction errors, and will yield similar spatial PSF variability in both epochs.

%%%%%%%%%%%%%%%%%%%%%%%%%%%%%%%%%%%%%%%%%%%%%%%%%%%%%%%%%%%%%%%%%%%%%%%%%%%

%   4. COORDINATED OBSERVATIONS
%       ((See https://hst-docs.stsci.edu/display/HSP/HST+Cycle+26+Preparation+of+the+PDF+Attachment)
%
%
\coordinatedobs          % Do not delete this command.
% Enter your coordinated observing plans here, if any.
\vspace{-0.5cm}
None.

%%%%%%%%%%%%%%%%%%%%%%%%%%%%%%%%%%%%%%%%%%%%%%%%%%%%%%%%%%%%%%%%%%%%%%%%%%%

%   5. JUSTIFY DUPLICATIONS
%       (See https://hst-docs.stsci.edu/display/HSP/HST+Cycle+26+Preparation+of+the+PDF+Attachment)
%
%
\duplications           % Do not delete this command.
% Enter your duplication justifications here, if any.
\vspace{-0.5cm}
We will observe all fields in two epochs to measure proper motions.
This use does not constitute a duplication.


%%%%%%%%%%%%%%%%%%%%%%%%%%%%%%%%%%%%%%%%%%%%%%%%%%%%%%%%%%%%%%%%%%%%%%%%%%%

\pagebreak
\printbibliography


\end{document}          % End of proposal. Do not delete this line.
                        % Everything after this command is ignored.

