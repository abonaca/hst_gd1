%%%%%%%%%%%%%%%%%%%%%%%%%%%%%%%%%%%%%%%%%%%%%%%%%%%%%%%%%%%%%%%%%%%%%%%%%%
%
%    phase1-GO.tex  (use only for General Observer and Snapshot proposals; use phase1-AR.tex for Archival Research and
%                      Theory proposals use phase1-DD.tex for GO/DD proposals and use phase1-MC.tex for GO/MC rapid response
%                       proposals).
%
%    HUBBLE SPACE TELESCOPE
%    PHASE I OBSERVING PROPOSAL TEMPLATE 
%     FOR CYCLE 25 and beyond
%
%    Version 2.0 April 2018 
%
%    Guidelines and assistance
%    =========================
%     Cycle 26 Announcement Web Page:
%
%         https://hst-docs.stsci.edu/ 
%
%    Please contact the STScI Help Desk if you need assistance with any
%    aspect of proposing for and using HST. Either send e-mail to
%    help@stsci.edu, or call 1-800-544-8125; from outside the United
%    States, call [1] 410-338-1082.
%
%
%%%%%%%%%%%%%%%%%%%%%%%%%%%%%%%%%%%%%%%%%%%%%%%%%%%%%%%%%%%%%%%%%%%%%%%%%%%

% The template begins here. Please do not modify the font size from 12 point.

\documentclass[12pt]{article}
\usepackage{phase1}

\usepackage{graphicx}
\usepackage{parskip} \setlength{\parindent}{0cm}

\newcommand{\HST}{\textsl{HST}}

\begin{document}

%   1. SCIENTIFIC JUSTIFICATION
%       (See https://hst-docs.stsci.edu/display/HSP/HST+Cycle+26+Preparation+of+the+PDF+Attachment)
%
%
\justification          % Do not delete this command.
% Enter your scientific justification here. 


\begin{figure}
\begin{center}
\includegraphics[width=\textwidth]{target_density.pdf}
\end{center}
\caption{
(\emph{Left}) Color-magnitude diagram of the GD-1 region in PanSTARRS bands (small gray points) and likely GD-1 members (large blue points), selected on Gaia proper motions and PS1 photometry as in Price-Whelan \& Bonaca (2018).
An old, metal-poor isochrone fits GD-1 at a distance of 8\,kpc (light blue line).
(\emph{Right}) Predicted number of stars in combined ACS/WFC and WFC3/UVIS fields-of-view as a function of magnitude (light blue line), scaled from the average density of GD-1 members on the bright end (dark blue line).
There are $\sim15$ member stars expected at $g\lesssim26$\,mag, which is observed at $\rm S/N\sim20$ in our setup.
Combining 2--4 fields, depending on the local stream density, will yield proper motions accurate to $\sim3\,\rm km\,s^{-1}$.
}
\label{fig:density}
\end{figure}

\begin{figure}
\begin{center}
\includegraphics[width=\textwidth]{pm_precision.pdf}
\end{center}
\caption{
Proper motion offset between the spur (pink) and the stream (purple) in the fiducial GD-1 model (solid crosses).
Model predictions convolved with final Gaia uncertainties are shown as faint error bars, while the \HST\ performance expected from this program is shown with filled circles.
\HST\ is the only facility that can resolve velocity offsets on the order of $10\,\rm km\,s^{-1}$ at the distance of the GD-1 stream (labeled dashed circles).
% , and thus constrain the perturber mass up to a factor of a few.
}
\label{fig:precision}
\end{figure}

%%%%%%%%%%%%%%%%%%%%%%%%%%%%%%%%%%%%%%%%%%%%%%%%%%%%%%%%%%%%%%%%%%%%%%%%%%%

%   2. DESCRIPTION OF THE OBSERVATIONS
%       (See https://hst-docs.stsci.edu/display/HSP/HST+Cycle+26+Preparation+of+the+PDF+Attachment)
%
%
\describeobservations   % Do not delete this command.
% Enter your observing description here.


%%%%%%%%%%%%%%%%%%%%%%%%%%%%%%%%%%%%%%%%%%%%%%%%%%%%%%%%%%%%%%%%%%%%%%%%%%%

%   3. SPECIAL REQUIREMENTS
%       (See https://hst-docs.stsci.edu/display/HSP/HST+Cycle+26+Preparation+of+the+PDF+Attachment)
%
%
\specialreq             % Do not delete this command.
% Justify your special requirements here, if any.
We are requesting coordinated parallel observations with WFC3.
GD-1 stream is wide enough to allow for simultaneous targeting with ACS and WFC3 ($\rm FWHM\sim0.5^\circ$, Grillmair \& Dionatos 2006).
These parallel observations will increase the number of observed GD-1 members by 60\,\% and thus improve measured velocities by 20\,\%.

For each field we request the same telescope orientation for both epochs.
This will maximize the overlapping area between the two epochs, will minimize differential geometric distortion correction errors, and will yield similar spatial PSF variability in both epochs.

% These parallel observations do not require special orientation constraints except that each orientation receives two separate visits so that we reach 2 orbit depth in each filter for each parallel field.

%%%%%%%%%%%%%%%%%%%%%%%%%%%%%%%%%%%%%%%%%%%%%%%%%%%%%%%%%%%%%%%%%%%%%%%%%%%

%   4. COORDINATED OBSERVATIONS
%       ((See https://hst-docs.stsci.edu/display/HSP/HST+Cycle+26+Preparation+of+the+PDF+Attachment)
%
%
\coordinatedobs          % Do not delete this command.
% Enter your coordinated observing plans here, if any.
None.

%%%%%%%%%%%%%%%%%%%%%%%%%%%%%%%%%%%%%%%%%%%%%%%%%%%%%%%%%%%%%%%%%%%%%%%%%%%

%   5. JUSTIFY DUPLICATIONS
%       (See https://hst-docs.stsci.edu/display/HSP/HST+Cycle+26+Preparation+of+the+PDF+Attachment)
%
%
\duplications           % Do not delete this command.
% Enter your duplication justifications here, if any.
We will observe all fields in two epochs to measure proper motions.
This use does not constitute a duplication.


%%%%%%%%%%%%%%%%%%%%%%%%%%%%%%%%%%%%%%%%%%%%%%%%%%%%%%%%%%%%%%%%%%%%%%%%%%%

\end{document}          % End of proposal. Do not delete this line.
                        % Everything after this command is ignored.

