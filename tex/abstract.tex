\documentclass[12pt]{article}

\begin{document}

Recently-released astrometric data from the Gaia mission (data release 2; DR2) has already begun to revolutionize our view of the Milky Way and its structure.
Our recent work using Gaia DR2 has demonstrated the utility of exquisite astrometry even for studying distant stellar structures such as stellar streams.
Thin stellar streams, formed from disrupted globular clusters, are a critical astrophysical probe of dark matter as they are sensitive to both the large-scale distribution of matter, and possible small-scale clustering of dark matter.
The existence of dark matter subhalos, and their mass spectrum, is one of the largest open questions in the path towards understanding and constraining dark matter models.
[Goal]
[Problem]
[TODO: plan with HST and strategy]
[Broader impacts]

\end{document}
